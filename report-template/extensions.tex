One way to enhance our back-end to LLVM-IR would be to make better use of the PHI function. We did not used them much, prefering to use pointers and access memory which is not SSA. This is not considered as a good practice in LLVM-IR.

Another interesting and useful extension would be to implement a garbage collector (mark and sweep or reference counting) to properly do automatic memory management.

A great advantage of generating LLVM-IR is to now dispose of a set of compiler from LLVM-IR to other languages espcially native code. Using "llc", the official LLVM-IR compiler or simply "clang" the C compiler from LLVM (which is able to compile LLVM-IR), we can obtain a machine-code binary for our operating system. Another great back-end compiler is "emscripten" which translate LLVM-IR to javascript and let us have Tool code running in our browsers! These two examples (binaries and javascript) are included in the examples directory of this report.

\section*{References}
LLVM Documentation: http://llvm.org/docs/ (last access: January 10, 2014)
