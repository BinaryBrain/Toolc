As an introductory example, a detailed compiled version of a factorial program written in Tool is given. As LLVM-IR, like other assembly languages, results in long code, this example is simple and not complete. Further difficulties like object-orientation or static single assignment will be detailed later in the text.

One Tool code for factorial function:
\begin{lstlisting}
def computeFactorial(num : Int) : Int = {
        var num_aux : Int;
        if (num < 0 || num == 1)
            num_aux = 1;
        else
            num_aux = num * 
		(this.computeFactorial(num - 1));
        return num_aux; }
\end{lstlisting}
and its compiled version (lines beginning with ";" are comments):
\begin{verbatim}
define i32 @Fact$computeFactorial(%struct.Fact* 
        %this, i32 %\_num) nounwind ssp
    
    ; Store the argument in a pointer
    ; %num is of type i32*
    %num = alloca i32, align 4
    store i32 %\_num, i32* %num, align 4

    ; var num_aux: Int
    %num\_aux = alloca i32, align 4

    ; %1 = num
    %1 = load i32* %num, align 4
    ; %3 = 0
    %2 = alloca i32, align 4
    store i32 0, i32* %2, align 4
    %3 = load i32* %2, align 4
    ; %4 = num < 0
    %4 = icmp slt i32 %1, %3
    br label %5
    
; <label>: %5
    ; Short-circuiting OR if num < 0
    br i1 %4, label %12, label %6
    
; <label>: %6
    ; %7 = num
    %7 = load i32* %num, align 4
    ; %9 = 1
    %8 = alloca i32, align 4
    store i32 1, i32* %8, align 4
    %9 = load i32* %8, align 4
    ; %10 = (num == 1)
    %10 = icmp eq  i32 %7, %9
    br label %11
    
; <label>: %11
    br label %12
    
; <label>: %12
    ; solve short-circuiting with PHI function
    ; that decides which value to give to %13
    ; based on the branch we are coming from
    %13 = phi i1 [ true, %5 ], [ %10, %11 ]

    ; if branching
    br i1 %13, label %14, label %17
    
; <label>: %14
    ; %16 = 1
    %15 = alloca i32, align 4
    store i32 1, i32* %15, align 4
    %16 = load i32* %15, align 4
    ; num_aux = 1
    store i32 %16, i32* %num\_aux, align 4
    br label %31
    
; <label>: %17
    ; %18 = num
    %18 = load i32* %num, align 4
    ; %20 = this
    %19 = alloca %struct.Fact*, align 8
    store %struct.Fact* %this,%struct.Fact** 
        %19, align 8
    %20 = load %struct.Fact** %19, align 8
    ; %21 = this
    %21 = load i32* %num, align 4
    ; %23 = 1
    %22 = alloca i32, align 4
    store i32 1, i32* %22, align 4
    %23 = load i32* %22, align 4
    ; %24 = num - 1
    %24 = sub nsw i32 %21, %23
    ; %26 = this.vtable
    %25 = getelementptr inbounds 
        %struct.Fact* %20, i32 0, i32 0
    %26 = load %struct.Fact$vtable** %25, 
        align 8
    ; %28 = this.vtable.computeFactorial
    %27 = getelementptr inbounds 
        %struct.Fact$vtable* %26,i32 0,i32 0
    %28 = load i32 (%struct.Fact*, i32)** 
        %27, align 8
    ; %29 = this.computeFactorial(num-1)
    %29 = call i32 %28(%struct.Fact* 
        %20, i32 %24)
    ; %30 = num_aux 
    ;   * this.computeFactorial(num-1)
    %30 = mul nsw i32 %18, %29
    ; update num_aux
    store i32 %30, i32* %num\_aux, align 4
    br label %31
    
; <label>: %31
    ; return num_aux
    %32 = load i32* %num\_aux, align 4
    ret i32 %32
\end{verbatim}
